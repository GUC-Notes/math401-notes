%% %%%%%%%%%%%%%%%%%%%%%%%%%%%%%%%%%%%%%%%%%%%%%%%%
%% Problem Set/Assignment Template to be used by the
%% Food and Resource Economics Department - IFAS
%% University of Florida's graduates.
%% %%%%%%%%%%%%%%%%%%%%%%%%%%%%%%%%%%%%%%%%%%%%%%%%
%% Version 1.0 - November 2019
%% %%%%%%%%%%%%%%%%%%%%%%%%%%%%%%%%%%%%%%%%%%%%%%%%
%% Ariel Soto-Caro
%%  - asotocaro@ufl.edu
%%  - arielsotocaro@gmail.com
%% %%%%%%%%%%%%%%%%%%%%%%%%%%%%%%%%%%%%%%%%%%%%%%%%

\documentclass[12pt]{article}
\usepackage{design_ASC}
\usepackage{amsmath}% http://ctan.org/pkg/amsmath

\theoremstyle{definition}
\usepackage{longtable}
\newtheorem{exmp}{Example}[section]
\newtheorem{slo}{Definition}[section]
\newcommand*{\Perm}[2]{{}^{#1}\!P_{#2}}%
\newcommand*{\Comb}[2]{{}^{#1}C_{#2}}%
\setlength\parindent{0pt} %% Do not touch this
\usepackage{amsmath}% http://ctan.org/pkg/amsmath
%% -----------------------------
%% TITLE
%% -----------------------------
\title{\textbf{Selected Problems}} %% Assignment Title

\author{\textbf{Ibrahim Abou Elenein}}

\date{\today} %% Change "\today" by another date manually
%% -----------------------------
%% -----------------------------

%% %%%%%%%%%%%%%%%%%%%%%%%%%
\begin{document}
\setlength{\droptitle}{-5em}    
%% %%%%%%%%%%%%%%%%%%%%%%%%%
\maketitle
% --------------------------
% Start here
% --------------------------
\section{Midterm  2017} 
% %%%%%%%%%%%%%%%%%%%
\subsection{Problem 4 a}
\begin{exmp}
    From a deck of 52 cards, a 5-card hand is dealt. Find the number of
    hands containing exactly one pair without considering the variation
    of the suits?

    \subsubsection{Answer}
    \textbf{Without considering the suits}, my approach would be:

    The pair can be one of 13 numbers, i.e. we need to pick 1 out of 13 candidates.

    For the remaining 3 cards,3 cards with different numbers. In this case we need to pick 3 out of 12 candidates. 
    \[
        \overbrace{\Comb{13}{1}}^{\text{number of ways to select one pair}}
        \times \underbrace{\Comb{12}{3}}_{\text{number of ways to select other 3 cards}}
    \]
\end{exmp}
\subsection{Problem 3 b}
\begin{exmp}
    In how many ways can we distribute 9 identical picses of candy
    to four children, if each child must get at least one piece?
    \subsubsection{Answer}
    The first four candies will be distributed to the children.
    Now we have 5 candies to be distributed to the childern
    Imagine 3 divisors (or blocks) we will use to group the candies,
    for each arrangement of the divisors and candies is a distribution.
    Now we have $8!$ arrangement but removing the duplicates for the
    candies $5!$ and $3!$ for the divisors 
    \[
        \frac{8!}{5!.3!}
    \]
\end{exmp}
\section{Midterm 16}
\subsection{Problem 3 b}
\begin{exmp}
    Three cards are drawn from a standard deck and lined on a table.
    Find the probability that the first (leftmost) card is a queen
    , the seconed is a jack, and the third (rightmost) is not a club.
    \subsubsection{Answer}
    First of all proving that choosing one queen or jack is independent of
    choosing not club 
    \[
        P(Q) = \frac{1}{13}; \ \ \ P(!C) = \frac{3}{4}
    \]
    \[
        P(!C \cap Q) = \frac{3}{52} = P(Q) \times P(!C)
    \]
So the Probability will be as follows 
\[
    \underbrace{P(Q)}_{\text{Probability of choosing queen}} 
    \times \underbrace{P(J|Q)}_{\text{Probability of choosing jack given queen}}
    \times \underbrace{P(!C)}_{\text{Probability of choosing not club}}
\]
\[
    \frac{1}{13}\   \ \ \ \ \  \ \ \ \ \ \ \ \ \ \times \ \ \ \ \ \ \ \ \ \  \ \ \ \
    \frac{3}{51} \ \ \ \ \ \ \ \ \ \  \ \ \ \ \times  \ \ \ \ \ \ \ \ \ \  \ \ \ \ \frac{3}{4}
\]
    
\end{exmp}
\end{document}
