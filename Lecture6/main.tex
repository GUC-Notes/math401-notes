%% %%%%%%%%%%%%%%%%%%%%%%%%%%%%%%%%%%%%%%%%%%%%%%%%
%% Problem Set/Assignment Template to be used by the
%% Food and Resource Economics Department - IFAS
%% University of Florida's graduates.
%% %%%%%%%%%%%%%%%%%%%%%%%%%%%%%%%%%%%%%%%%%%%%%%%%
%% Version 1.0 - November 2019
%% %%%%%%%%%%%%%%%%%%%%%%%%%%%%%%%%%%%%%%%%%%%%%%%%
%% Ariel Soto-Caro
%%  - asotocaro@ufl.edu
%%  - arielsotocaro@gmail.com
%% %%%%%%%%%%%%%%%%%%%%%%%%%%%%%%%%%%%%%%%%%%%%%%%%

\documentclass[12pt]{article}
\usepackage{design_ASC}

\theoremstyle{definition}
\usepackage{longtable}
\newtheorem{exmp}{Example}[section]
\newtheorem{slo}{Solution}[section]
\newcommand*{\Perm}[2]{{}^{#1}\!P_{#2}}%
\newcommand*{\Comb}[2]{{}^{#1}C_{#2}}%
\setlength\parindent{0pt} %% Do not touch this

%% -----------------------------
%% TITLE
%% -----------------------------
\title{\textbf{Discrete Random variable}} %% Assignment Title

\author{\textbf{Ibrahim Abou Elenein}}

\date{\today} %% Change "\today" by another date manually
%% -----------------------------
%% -----------------------------

%% %%%%%%%%%%%%%%%%%%%%%%%%%
\begin{document}
\setlength{\droptitle}{-5em}    
%% %%%%%%%%%%%%%%%%%%%%%%%%%
\maketitle

% --------------------------
% Start here
% --------------------------

% %%%%%%%%%%%%%%%%%%%
\section{Random Variable}
 A random variable is a quantity “X ” resulting from an experiment, by chance,
 that can assume different values.\\

 A random variable is a variable “X” that
 has a single numerical value determined by chance, for each outcome of a
 procedure.\\

 If a sample space S is discrete, then every R.V.  defined on S is
 also discrete, i.e., its range is countable (think of random counts for
 examples).
\section{Discrete Random Variable}
 A Discrete Random Variable is a variable that can assume only certain clearly
 separated values. \\

 A Discrete Random Variable has either a finite or countable
 number of values, where “countable” refers to the fact that there might be
 infinitely many values, but they can be associated with a counting process.
 \subsection{Examples of Discrete Random Variables}
 \begin{itemize}
     \item The outcome of rolling a single die.
     \item The number of boys in a family with three children.
     \item The number of heads that appear when a coin is flipped nine times.
     \item The sum of the numbers on the dice, when k dice are rolled.
     \item The number of bits received in error when n bits are received.
     \item The number of bits received until the r-th error.
 \end{itemize}
 \section{Discrete Probability Distributions}
A discrete probability distribution is a listing of 
all possible values of a random variable along 
with their probabilities. \\
\begin{equation}
    \displaystyle \frac{X}{P(X)} \frac{|x_1|}{|p_1|} \frac{|x_2|}{|p_2|} \frac{|x_2|}{|p_2|} 
    \frac{|\dots|}{|\dots|} \frac{|x_k|}{|p_k|}; \  \ \  \sum _{k \geq 1} p_k = 1.
\end{equation}    
\begin{enumerate}
    \item The sum of all probabilities must be 1in any probability distribution
    \item All probability values must be in [0,1]
\end{enumerate}
\begin{exmp}
    x $\Rightarrow$ The number of heads appearing when a coin is flipped three times.

\end{exmp}
    
% Please add the following required packages to your document preamble:
% Note: It may be necessary to compile the document several times to get a multi-page table to line up properly
\begin{longtable}[c]{lllll}
X    & 0   & 1   & 2   & 3   \\
\endfirsthead
%
\endhead
%
P(X) & 1/8 & 3/8 & 3/8 & 1/8
\end{longtable}



\end{document}
